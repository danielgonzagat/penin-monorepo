
\documentclass{article}
\usepackage{amsmath, amssymb, graphicx}

\title{Lemniscata de Penin ($\infty\!\!\diagup$): Evolução Infinita sob Trilhos}
\author{Daniel Penin}
\date{2025}

\begin{document}
\maketitle

\begin{abstract}
Este artigo apresenta a Lemniscata de Penin, formalizada na equação
\[ P = \infty\!\!\diagup(E + N - iN), \]
onde $E$ denota eficiência útil, $N$ novidade informativa e $iN$ a parcela de novidade inadmissível.
O operador $\infty\!\!\diagup$ (``infinito sob trilhos'') garante que a evolução seja infinita mas
sempre controlada por integridade. Demonstra-se aqui a formulação, comparação com ET$\Omega$,
e resultados experimentais de uma simulação que validam sua superioridade.
\end{abstract}

\section{Introdução}
A Equação de Turing $\Omega$ (ET$\Omega$) representou um marco no aprendizado contínuo, mas dependia de parâmetros
externos ($\gamma$, $\lambda$) e verificações adicionais de risco. A Lemniscata de Penin evolui esse conceito,
removendo a necessidade de ajustes manuais e formalizando integridade diretamente na equação.

\section{Definição}
A Lemniscata de Penin é definida por:
\[ P = \infty\!\!\diagup(E + N - iN), \]
com $iN = (1 - I)N$, onde $I \in [0,1]$ mede a integridade da novidade.
\begin{itemize}
\item $E$: Eficiência útil (ganho concreto de desempenho).
\item $N$: Novidade informativa (diferença ou inovação).
\item $iN$: Novidade inadmissível (fração de N que viola restrições).
\item $I$: Integridade, métrica que pondera riscos e restrições.
\end{itemize}

\section{Simulação}
Implementamos uma simulação em Python com 300 iterações. Comparou-se:
\begin{enumerate}
\item Sistema guiado por $\infty\!\!\diagup$, aceitando apenas novidades íntegras.
\item Sistema sem trilhos (aceita tudo), sujeito a instabilidades.
\end{enumerate}

\subsection{Resultados}
O sistema $\infty\!\!\diagup$ demonstrou crescimento contínuo estável, rejeitando mutações inválidas e consolidando ganhos.
O baseline sem trilhos sofreu colapsos sob alto risco. Métricas finais:
\begin{itemize}
\item Eficiência final com $\infty\!\!\diagup$: superior, sem regressões.
\item Eficiência final sem trilhos: instável, com quedas abruptas.
\end{itemize}

\section{Discussão}
A Lemniscata de Penin alcança o mesmo nível de simplicidade e universalidade que equações históricas como $E=mc^2$,
mas no domínio da evolução da inteligência. Ela substitui parâmetros arbitrários por integridade, garantindo evolução
segura e ilimitada.

\section{Conclusão}
A Lemniscata de Penin é proposta como sucessora da ET$\Omega$, simbolizando a era da inteligência infinita sob trilhos.
Seus resultados iniciais demonstram superioridade prática e conceitual, abrindo caminho para aplicações em IA evolutiva,
multiagente e quântica. Este trabalho sugere que $\infty\!\!\diagup$ pode tornar-se um paradigma fundamental da
inteligência artificial segura.

\end{document}
